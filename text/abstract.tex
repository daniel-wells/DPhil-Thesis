% "not normally exceeding 300 words"

% 1/2 sentences basic into comprehensible to any scientist
Meiosis is a specialised cell division producing the haploid gametes required for sexual reproduction.
A key function of this process is to generate genetic diversity through recombination and independent assortment of homologous chromosomes.
Delineation of gene expression during this process has been challenging due to the heterogeneity of testis tissue and the lack of faithful \textit{in vitro} models.

%The mechanisms of meiosis have important implications in the fertility, speciation, and evolution of all sexually reproducing organisms.

% One sentence summarying result, "here we show"
Here we jointly analyse a single-cell resolution transcriptomic dataset of over 20,000 cells from both wild type and mutant mice.
We use dimensionality reduction methods to infer latent components of variation that represent transcriptional programmes, mutant specific pathological processes, and technical effects.
This approach simultaneously soft clusters cells, infers corresponding groups of co-expressed marker genes, and imputes sparse, noisy gene expression.
We were also able to infer, \textit{de novo}, transcription factor binding motifs for each component, revealing a general switch at the meiotic divisions.
We facilitate access to this resource by providing an interactive website \href{http://www.testisatlas.ml}{testisatlas.ml}.

%We also infer, \textit{de novo}, transcription factor binding motifs likely to be driving the patterns of co-expression we observe.

The high-resolution delineation of gene expression during the spermatogenic programme provides high-resolution clues to the function of understudied genes.
One such gene, \textit{Zcwpw1}, is highly co-expressed with \textit{Prdm9}, and has domains capable of recognising the unique combination of histone marks that PRDM9 deposits (H3K4me3 and H3K36me3).
By using a human \textit{in vitro} system of \textit{Zcwpw1} co-transfection with either human or chimp \textit{Prdm9}, we show that PRDM9 causes the recruitment of ZCWPW1 to its binding sites.
This recruitment is stronger than for sites with H3K4me3 alone and is CpG dependent.

Male \textit{Zcwpw1\textsuperscript{-/-}} mice have completely normal double strand break positioning, but severe repair and synapsis defects leading to complete testicular azoospermia.
Although PRDM9's effect of DSB positioning remains intact, PRDM9's effect of aiding synapsis appears to be abolished, with persistent DMC1 signal - most dramatically at the most strongly PRDM9 bound hotspots.

% Two or three sentences explaining what the main result reveals in direct comparison to what was thought to be the case previously, or how the main result adds to previous knowledge


%One or two sentences to put the results into a more general context

%Our results demonstrate

% Two or three sentences to provide a broader perspective, readily comprehensible to a scientist in any discipline, may be included in the first paragraph if the editor considers that the accessibility of the paper is significantly enhanced by their inclusion

%We anticipate this expression atlas
